% Datei: USERDISK_3:[FLECHSIG.PHASE.TEXT]USERAPI.TEX
% Datum: 22.JUN.1995
% Stand: 18-MAR-1996
% Autor: FLECHSIG, BESSY Berlin

\chapter{User Interface}
The graphical user interface is mostly self explaining. 
\section {Main Program}
\subsection{Start}
After \prog {@phaseinit} the program can be started with \prog {phase}. Then a
window for graphical output and the main window (figure \ref{mw}) and an
information window will appear.
\begin{figure}
\centerline{ \hbox{
 \psfig{figure=pmain.eps,height=4cm,angle=270}}}  
  \caption {\label{mw} \phase main Window with status messages} 
\end {figure}     

\subsection {Prinziple}
To make an calculation one has to generate a phase space transformation matrix
for the optical system. The matrix can describe one optical element or may be a
product matrix of different elements.
For one element do as follows:  
\begin{enumerate} 
\item select a fileset for the element using the file/files dialog (figure 
\ref{fseldia}).
\item create a geometry set with the edit/geometry dialog.
\item create a optical element set with the edit/optical element dialog.    
\item create a source with the edit/source dialog.    
\item create a matrix with commands/create matrix.
\item make the calculation with commands/calculation
\item graphical output with commands/graphic
\end {enumerate}

\begin{figure}
\centerline{ \hbox{
 \psfig{figure=fileseldialog.eps,height=9cm}}}  
  \caption {\label{fseldia} File selection Box, each file can be selected
pressing the button, pressing the icon button selects default fileneames
according to the matrix file.} 
\end {figure}  
   
For optical systems on has to generate the matrixes for each element. In a next
step the transformation matrix of the whole system has to be generated using
the commands/multiply matrix dialog box \ref {mbox}. Then this matrix has to be
included in a fileset and a source has to be selected after that the
calculation can be carried out directly. (!! do never press commands/create 
matrix - probably you would not overwrite the product matrix!!). 

\begin{figure}
\centerline{ \hbox{
 \psfig{figure=mamubox.eps,height=5cm,angle=270}}}  
  \caption {\label{mbox} File selection Box, each file can be selected
pressing the button, pressing the icon button selects default fileneames
according to the matrix file.} 
\end {figure}    



\section{Graphical Output}
A direct graphical output is implemented using a very limited selection
of the HIGZ/HPLOT package from the CERN library. For example: scatter plots,
lego plots, surface plots, contour plots. Similar to the RAY program
footprints, vertical and horizontal profiles are included. With tight
boundarys and profile plots  "cuts" are possible.
For more sophisticated output
a direct usage of PAW, IDL, GNUPLOT $\ldots$ is recommended.


\section {Filesystem}

\subsection {Ray Type}
 
For ray trace input and output files of the "ray type" are used. The 
format is:

\%i 0                  header: $<number of rays> <0>$; 0= ray filetype \\
\%lf \%lf \%lf \%lf       ray : $<y> <z> <dy> <dz> $        \\
 $\vdots$                                                  \\
\%lf \%lf \%lf \%lf        \\

With the  '0' in the header  \phase recognizes the file as ray type.
 

{\it filename: userdisk\_3:[flechsig.phase.text]userapi.tex}  

